\chapter{Introduction}

\section{Motivation}
Why do we make the research? 
Requirements of the Semantics Web Research. 
Current solutions.
The main research papers that are going to be used as the basis for the study.
There is no definite guide for the software (feel the gap between the detailed
documentation and superficial general-purpose manuals).

\section{The OpenLink Virtuoso System}
The general info about the Virtuoso System. It's properties and applications -
general description.

\section{Objectives}
\label{sec:Objectives}
The main objective of the thesis\cite{6079245} is to analyse and describe the
properties of the Virtuoso System that are crucial for the research over the
semantic webs.

\chapter{Problem analysis}
\label{chap:Problemanalysis}

\section{Semantic Web}

\subsection{Definitions}
Semantic Web is a project focused on definition and publication of the standards
regarding content descriptions on the Internet. The main objective of these
standards is to provide the content in a form convenient for the effective
information processing by the software and hardware. The semantic Web
standards include OWL (\textit{Web Ontology Language}), RDF (\textit{Resource
Description Framework}, Sec.~\ref{sec:RDF}) and RDFS (\textit{RDF Schema}).
The meanings of the informational resources are defined using
ontologies (Sec.~\ref{sec:Ontology}) - this representation are discussed in
details in the next section.

Semantic Web was an idea of Tim Berners-Lee who is the chef of W3C, the creator
of the WWW standard and the first Web browser. In principle, semantic Web should
be based on the existing communication protocols that set up the contemporary
Internet. The main difference is that the published data must be also
\textit{comprehensible} for the machines. To achieve this objective, the
resources are presented in a form that allows to identify their context
and the relations between them.

\subsection{Ontologies}
\label{sec:Ontology}
In general, ontology is a formal representation of a knowledge domain, that
is composed of sets of concepts and relations between them. This system creates
a conceptual schema that provides a description of some domain. In addition,
the conceptual schema can be used as a basis to draw conclusions about the
properties of the terms described by a given ontology. According to
XYZ, Ontologies are classified into lightweight ontologies and heavylight
ontologies according to it expressiveness [4]. The lightweight ontology, sometimes
called terminology, is simple a taxonomic structure of concepts
and some includes simple comments on relations between
concepts. The heavyweight ontology extensively axiomatizes
concepts and relations to represent ontological commitment
explicitly and it is thus composed of concepts, relations and
rules. Every heavyweight ontology can have a lightweight
version. And there is not a clear borderline between light to
heavy weight.


\subsection{Field of study}

\subsection{Research trends}

\subsection{Data Storage}

\section{Data Storage Formats}
\subsection{Relational DBs}

\subsection{Resource Description Framework}
\label{sec:RDF}

\section{Overview of existing systems}
\ldots

\subsection{Commercial Platforms}
\label{sec:commercial_platforms}

\subsubsection{Platform 1}
\ldots

\subsection{OpenSource Platforms}
\label{sec:open_source_platforms}

\subsubsection{Platform 11}
\ldots

\section{General properties of the Virtuoso System}
\subsection{History of development}
\subsection{Properties of Virtuoso}
\subsection{Main applications}
\subsection{Aspects important for the Semantic Webs}


\section{Used Tools}


\subsection{LUBM Benchmark}
The Lehigh University Benchmark is developed to facilitate the evaluation of
Semantic Web repositories in a standard and systematic way. The benchmark is
intended to evaluate the performance of those repositories with respect to
extensional queries over a large data set that commits to a single realistic
ontology. It consists of a university domain ontology, customizable and
repeatable synthetic data, a set of test queries, and several performance metrics.

\chapter{System analysis}
\section{Internal Structure}
\subsection{SPARQL end-point}
\subsection{Advantages}
\subsection{Special queries and commands}

\section{Key Features}
\subsection{Reasoning}
Forward/backward chaining.

\subsection{Import Data Mechanisms}

\subsection{Export Data Mechanisms}

\section{Comparison with similar systems}
\subsection{Virtuoso vs Oracle NoSQL}

\chapter{Conclusions}
The section includes the following:
 Overall analysis and integration of the research and conclusions of the thesis in light of current research in the field
 Conclusions regarding goals or hypotheses of the thesis that were presented in the Introduction, and the overall significance and contribution of the thesis research
 Comments on strengths and limitations of the thesis research
 Discussion of any potential applications of the research findings
 An analysis of possible future research directions in the field drawing on the work of the thesis
\ldots