\chapter{Introduction}

\section{Motivation}
Why do we make the research? 
Requirements of the Semantics Web Research. 
Current solutions. 
The main research papers that are going to be used as the basis for the study.
There is no definite guide for the software (feel the gap between the detailed
documentation and superficial general-purpose manuals).

\section{The OpenLink Virtuoso System}
The general info about the Virtuoso System. It's properties and applications -
general description.

\section{Objectives}
\label{sec:Objectives}
The main objective of the thesis\cite{6079245} is to analyse and describe the
properties of the Virtuoso System that are crucial for the research over the
semantic webs.

\chapter{Problem analysis}
\label{chap:Problemanalysis}

\section{Semantic Webs}
\ldots

\subsection{Field of study}

\subsection{Research trends}

\subsection{Data Storage}

\section{Data Storage Formats}
\subsection{Relational DBs}
\subsection{RDF}

\section{Overview of existing systems}
\ldots

\subsection{Commercial Platforms}
\label{sec:commercial_platforms}

\subsubsection{Platform 1}
\ldots

\subsection{OpenSource Platforms}
\label{sec:open_source_platforms}

\subsubsection{Platform 11}
\ldots

\section{General properties of the Virtuoso System}
\subsection{History of development}
\subsection{Main applications}
\subsection{Aspects important for the Semantic Webs}

\section{Used Tools}
\subsection{LUBM Benchmark}

\chapter{System analysis}
\section{Internal Structure}
\subsection{Reasoning}
Forward/backward chaining.

\subsection{SPARQL end-point}

\section{Comparison with similar systems}
\subsection{Virtuoso vs Oracle NoSQL}

\chapter{Conclusions}
\ldots