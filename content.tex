\chapter{Introduction}

\section{Motivation}
Why do we make the research? 
Requirements of the Semantics Web Research. 
Current solutions.
The main research papers that are going to be used as the basis for the study.
There is no definite guide for the software (feel the gap between the detailed
documentation and superficial general-purpose manuals).

\section{The OpenLink Virtuoso System}
The general info about the Virtuoso System. It's properties and applications -
general description.

\section{Objectives}
\label{sec:Objectives}
The main objective of the thesis is to analyse and describe the
properties of the Virtuoso System that are crucial for the research over the
semantic webs.

\chapter{Problem analysis}
\label{chap:Problemanalysis}

\section{Semantic Web}

\subsection{Definitions}
Semantic Web is a project focused on definition and publication of the standards
regarding content descriptions on the Internet. The main objective of these
standards is to provide the content in a form convenient for the effective
information processing by the software and hardware. The semantic Web
standards include OWL (\textit{Web Ontology Language}), RDF (\textit{Resource
Description Framework}, Sec.~\ref{sec:RDF}) and RDFS (\textit{RDF Schema}).
The meanings of the informational resources are defined using
ontologies (Sec.~\ref{sec:Ontology}) - this representation are discussed in
details in the next section.

Semantic Web was an idea of Tim Berners-Lee who is the chef of W3C, the creator
of the WWW standard and the first Web browser. In principle, semantic Web should
be based on the existing communication protocols that set up the contemporary
Internet. The main difference is that the published data must be also
\textit{comprehensible} for the machines. To achieve this objective, the
resources are presented in a form that allows to identify their context
and the relations between them.

\subsection{Ontologies}
\label{sec:Ontology}
In general, ontology is a formal representation of a knowledge domain, that
is composed of sets of concepts and relations between them. This system creates
a conceptual schema that provides a description of some domain. In addition,
the conceptual schema can be used as a basis to draw conclusions about the
properties of the terms described by a given ontology. 

According to Yu Juan and Dang Yanzhong\cite{5592098}, ontologies are classified
into two groups, according to their expressiveness:
\begin{itemize}
  \item lightweight ontologies
  \item heavylight ontologies
\end{itemize}

The lightweight ontologies, also referred as terminologies, are simple taxonomic
structures of concepts with simplified relationships between them. On the
other hand, the heavyweight ontologies  are composed of concepts, relations and
rules that represent ontological commitment explicitly. Sometimes is difficult
to clearly assign an ontology to one of these types.

\subsection{Field of study}

\subsection{Research trends}

\subsection{Data Storage}

\section{Data Storage Formats}
\subsection{Relational DBs}

\subsection{Resource Description Framework}
\label{sec:RDF}

\section{Overview of existing systems}
\ldots

\subsection{Commercial Platforms}
\label{sec:commercial_platforms}

\subsubsection{Platform 1}
\ldots

\subsection{OpenSource Platforms}
\label{sec:open_source_platforms}

\subsubsection{Platform 11}
\ldots

\section{General properties of the Virtuoso System}
\subsection{History of development}
\subsection{Properties of Virtuoso}
\subsection{Main applications}
\subsection{Aspects important for the Semantic Webs}


\section{Used Tools}

\subsection{Installation of Virtuoso}
The analysis was based on the newest stable version (ver. 6.1.5) of the
OpenLink Virtuoso compiled and installed from the source code obtained from the
SourceForge project page (ref.~\cite{VirtuosoUrl}). The software was installed
on the following machine:
\begin{itemize}
  \item Operating System: Linux Ubuntu 12.04 LTS 32-bit version
  \item Processor: Intel\textregistered Core\texttrademark2 Duo CPU T7500 @
  2.20GHz × 2
  \item RAM: 2 x 2 GB SODIMM DDR2 Synchronous 667 MHz 
\end{itemize}

In addition, the Virtuoso was configured with the following parameters
(settings in the \texttt{virtuoso.ini} file):

\begin{itemize}
  \item \texttt{NumberOfBuffers = 170000} (what corresponds to about 1410 MB of
  RAM)
  \item \texttt{MaxDirtyBuffers = 130000}
  \item \texttt{MaxCheckpointRemap = 42500} (25\% of \texttt{NumberOfBuffers})
\end{itemize}

\subsection{The LUBM Benchmark}
The study required a sample RDF data to be stored and processed by Virtuoso. For
this purpose, the LUBM Benchmark (\textit{The Lehigh University Benchmark},
ref.~\cite{LUBMUrl}) was used. According to its creators, LUBM was developed to
facilitate the evaluation of Semantic Web repositories in a standard and
systematic way. The tool evaluates the performance of those repositories with
respect to extensional queries over a large data set that commits to a single
realistic ontology. 

The sample data used in the benchmark is an ontology describing a university
structure. It can be dynamically generated in a customizable and repeatable
way. In addition, it includes a set of test queries and provides several
performance metrics.

To start with, the newest benchmark code (version 1.7) has been downloaded from
the LUBM page: \cite{LUBMUrl}. Apart from this, it is necessary to apply a
special patch that makes the tool work with Linux paths. Then, the
testing ontology data was generated for three universities with the following
command:

\begin{verbatim}
java -cp classes edu.lehigh.swat.bench.uba.Generator -univ 3 \
-index 0 -seed 1234 \
-onto http://www.lehigh.edu/~zhp2/2004/0401/univ-bench.owl
\end{verbatim}

According to the above \texttt{-onto} parameter, the appropriate ontology was
downloaded \cite{LUBMUrlOWL}. Apart from this, the
\texttt{VirtBulkRDFLoaderScript.vsql} script was prepared and loaded via
\texttt{isql} to enable the bulk loading of all generated ontologies.

\chapter{System analysis}
\section{Internal Structure}
\subsection{SPARQL end-point}
\subsection{Advantages}
\subsection{Special queries and commands}

\section{Key Features}
\subsection{Reasoning}
Forward/backward chaining.
In essence, Virtuoso's support for subclasses and subproperties is backward
chaining, i.e. it does not materialize all implied triples but rather looks for
the basic facts implying these triples at query evaluation time.

\subsection{Import Data Mechanisms}

\subsection{Export Data Mechanisms}

\section{Comparison with similar systems}
\subsection{Virtuoso vs Oracle NoSQL}

\chapter{Conclusions}
The section includes the following:
 Overall analysis and integration of the research and conclusions of the thesis in light of current research in the field
 Conclusions regarding goals or hypotheses of the thesis that were presented in the Introduction, and the overall significance and contribution of the thesis research
 Comments on strengths and limitations of the thesis research
 Discussion of any potential applications of the research findings
 An analysis of possible future research directions in the field drawing on the work of the thesis
\ldots